\section{Social media analytics}
Questo campo si è sviluppato dopo gli anni 2000, in seguito all'utilizzo dei social media: è difficile ottenere dati che descrivano esaustivamente una popolazione, quindi si ricorre al monitoraggio degli eventi in una community.

I social media si sono diffusi per avere informazioni in real time, sviluppare relazioni o strategie di marketing tramite interazioni tra provider e utenti. 

Le applicazioni delle analitiche sono per esempio information retrieval per estrarre e riassumere contenuti, filtering di comportamenti (hate speech) o caratterizzazione di topic.

L'informazione è puramente una collezione di fatti riguardo un'attività; al contrario, la conoscenza è la comprensione e consapevolezza dei fatti per poter prendere decisioni. Distinguere queste categorie è importante per poter identificare e trasformare dati.

\subsection{Natural Language Processing}
Essendo la quantità di informazioni molto grandi, è necessario effettuare una riduzione. La prima divisione è tra dato strutturato e non strutturato, partendo dal testo e selezionando le features interessanti. 

La tecnica che permette di passare dai dati non strutturati a strutturati è il Natural Language Processing: esso si riferisce alla capacità di un sistema di processare parole, frasi e documenti in un contesto di linguaggio naturale. Le frasi sono positive, negative o neutre a seconda dell'emozione che trasmettono (polarità).

Vengono collezionati tutti i dati dai social media provider, per poi rappresentarli in un modo comprensibile per l'elaboratore e classificarli. Il processo dev'essere veloce dato che le opinioni in tempo reale sono in costante variazione.

L'obiettivo è primariamente la comprensione del linguaggio da un computer (intelligenza artificiale), e conseguentemente lo studio del funzionamento della comunicazione umana (linguistica).

\begin{itemize}
	\item Opinion-holder: l'utente che detiene un'opinione su un argomento;
	\item Oggetto: ciò a cui si riferisce l'opinione;
	\item Aspetto. l'aspetto specifico dell'oggetto;
	\item Opinione: la visione dell'utente;
	\item Social network: la rete in cui viene condivisa l'informazione.
\end{itemize}

Inizialmente i sistemi NLP si basavano su input grammaticalmente corretti, senza errori di battitura e in un unico linguaggio.

Gli algoritmi moderni riescono a rimediare anche a queste problematiche, ma c'è ancora difficoltà nella rimozione del rumore e delle ambiguità (comuni nei social). 

L'ambiguità si rimedia tramite named-entity recognition o linking: esistono modelli probabilistici in grado di riconoscere le named-entities, per poi attribuire loro un significato tra i possibili. L'ironia e il sarcasmo vengono individuate tramite irony detection. 

Dato un insieme di testi, si crea un vocabolario: ogni parola corrisponde a una cella in un vettore, riempito con 0 o 1 se la parola compare. Il preprocessamento consiste nello stemming, cioè la rimozione dei suffissi.
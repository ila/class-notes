\section{Data analytics issues}

\subsection{Istanze, classi e attributi}
\textbf{Istanza} (oggetto, record): esempio descritto da un numero di attributi. \\
\textbf{Attributo} (campo, caratteristica): misura gli aspetti delle istanze. \\
\textbf{Classe}: gruppo di istanze.

Le istanze sono tipi specifici di esempi che devono essere classificati, associati ed eventualmente raggruppati. Possono essere \textit{dipendenti} o \textit{indipendenti} e sono caratterizzate da un numero predeterminato di attributi: l'input ai modelli di apprendimento sono istanze contenute in un dataset, basandosi su assunzioni \textbf{IID} (indipendenti e identicamente distribuite, derivano dalla stessa distribuzione).

La rappresentazione in forma proposizionale (tabellare) implica la definizione degli attributi, rimuovendo le relazioni ed esprimendole eventualmente tramite campi e condizioni. Viene considerato solo l'insieme delle osservazioni disponibili (``mondo chiuso"). 

Il processo di flattening di relazioni per creare un'unica tabella si indica con \textbf{proposizionalizzazione}: è possibile con qualsiasi insieme finito, ma causa ``biased models" con irregolarità o dati replicati rispetto al modello originale. Le relazioni $1 : n$ sono gestite associando un campo aggiuntivo nel lato 1 oppure con matrici booleane. L'operazione di \textit{join} generalmente causa rappresentazioni non totalmente veritiere (distorsioni), ma che hanno minore complessità computazionale.

Gli attributi sono le \textit{features} che costituiscono lo spazio di rappresentazione dell'input: devono sempre avere lunghezza predefinita, eventualmente si ricorre all'uso di flag. L'esistenza di alcuni attributi (derivabili) può dipendere da altri, e questo potrebbe aumentare la complessità del modello. 

Le variabili \textbf{nominali} (simboliche, categoriche, discrete) rappresentano una quantità nominale e non hanno relazioni logico-matematiche tra loro. \\
Le variabili \textbf{ordinali} (numeriche) impongono un ordine (numeri, stringhe). \\
Spesso non è immediata la distinzione tra nominali e ordinali. 

Conoscere la natura dell'attributo è essenziale per poter effettuare confronti e avere un criterio per le operazioni, trattare i dati mancanti e gestire le problematiche legate alla qualità dei dati.

\subsection{Data analytics tasks}
\subsubsection{Classification learning}
Supervisionato, si occupa della classificazione di campioni predefiniti in classi, secondo approcci di machine learning (regressioni, alberi di decisione, reti bayesiane). 

Il modello deve avere buone capacità di generalizzazione per input mai trattati prima, ed è possibile definire modelli di apprendimento in base a regole logiche su rappresentazioni proposizionali. Le regole logiche sono codificate usando \textit{if}.

\subsubsection{Clustering}
Il clustering serve per identificare gruppi di istanze simili. Gli algoritmi sono non supervisionati: la classe di un esempio non è conosciuta in partenza. Vengono utilizzati per la segmentazione.

\subsubsection{Associazione}
Modello predittivo non supervisionato con l'obiettivo della comprensione di associazioni: dall'esistenza di un attributo prevedere l'esistenza di un altro.

\subsubsection{Predizione numerica}
Supervisionato, modelli con un valore target in input: cerca di individuare relazioni tra attributi numerici (regressione).

\section{Data preprocessing}
I dati solitamente contengono problematiche che devono essere affrontate prima di poterli dare in input al modello: il preprocessing è un'attività fondamentale per individuare rumore, inconsistenze e incorrettezze.

Il processo di \textbf{data cleaning} si occupa di rimpiazzamento di valori mancanti e smoothing dei dati rumorosi. Ci sono modelli in grado di gestire per natura i missing values, ma altri hanno necessariamente bisogno della completezza. Non tutti i dati incompleti possono essere sostituiti.
\begin{itemize}
	\item MCAR (Missing Completely At Random): lla distribuzione di un esempio con valori mancanti non dipende da altri attributi;
	\item MAR (Missing At Random): la distribuzione di un esempio con valori mancanti dipende dagli attributi osservati, non necessariamente mancanti;
	\item NMAR (Not Missing At Random): la distribuzione di un esempio con valori mancanti dipende da attributi con valori mancanti.
\end{itemize}

I dati mancanti si possono ignorare, convertire a valori di default o rimpiazzare. Alcune tecniche di sostituzione implicano l'utilizzo della media (per valori continui con distribuzione normale) o della moda (discreti). Un altro modo è k-NN, che associa la classe sulla base della maggioranza degli oggetti vicini.

Un metodo di discretizzazione (smoothing) è il binning: divide il range in $N$ intervalli in base alla media (distribuzioni normali) o alla frequenza (skewed).



\newpage
\section{Ripasso strutture dati}

\subsection{Array}
Gli array sono strutture statiche con memoria allocata a priori. Sono veloci perché tutta la memoria viene allocata nello stesso blocco.

\subsection{Liste dinamiche}
Le liste dinamiche hanno con memoria dinamica che varia durante l'esecuzione (la memoria non è allocata nella stessa locazione spaziale, quindi sono più lente). \par
Esempi di liste dinamiche sono le linked list: hanno una $head$ che punta a una casella contenente $key$ e $next$, il quale a sua volta punta all'elemento successivo. Seguendo i puntatori si può visitare la lista. \\
Il tempo di accesso ai singoli dati non è costante, ma dipende dalla lunghezza della lista (inefficiente). La gestione dei puntatori è a carico del programmatore, quindi possono esserci errori anche gravi. \par 
Può essere utile immagazzinare anche il puntatore al valore precedente $previous$, per operazioni come la cancellazione. In questo caso la struttura dati si definisce lista doppia. \par
Un altro puntatore statico è $tail$, all'ultima casella. Serve per aggiungere degli elementi in ordine senza dover scorrere tutta la lista e va aggiornato dopo ogni inserimento. \par 
Se l'ultima casella punta alla prima e la prima ha come $previous$ l'ultima la lista è doppia circolare, e non è necessario avere $tail$.
A volte si definisce un valore sentinella, una casella "finta" per gestire alcuni sotto-casi particolari (lista vuota, ultimo elemento, \dots).

\subsection{Alberi radicati}
Gli alberi radicati (generalmente binari) hanno puntatori a $left$ e $right$, che indicano i rami del nodo. Il campo $key$ contiene il valore.

\subsection{Alberi binari di ricerca}
Gli alberi binari di ricerca (BST) sono particolari alberi che rispettano la seguente proprietà: nel sottoalbero sinistro a ogni nodo ci sono solo valori inferiori, e nel sottoalbero destro ci sono solo valori superiori. Il tempo computazionale delle operazioni sui BST è di solito logaritmico.

\subsection{Heap}
Un heap è un array rappresentabile come albero. Tutti i livelli sono riempiti tranne l'ultimo, che viene riempito da sinistra verso destra. L'ordinamento di un heap si effettua con la visita in-order. Eventualmente viene aggiunto un puntatore $parent$ al padre di ogni foglia. \par
Il max heap è un tipo particolare di heap dove la radice corrisponde al massimo.

\subsection{Pile}
Le pile (stack) possono essere implementate con array e liste. Sono gestite con politica LIFO, l'ultimo elemento è il primo a essere rimosso. Le operazioni su pile sono $push$ e $pop$, e $stackempty$ per evitare errori di overflow/underflow. Per leggere l'elemento in cima si usa $top$.

\subsection{Code}
Le code (queue) possono essere implementate con array e liste. Sono gestite con politica FIFO, il primo elemento è il primo a essere rimosso. Le operazioni su code sono $enqueue$ e $dequeue$, $emptyqueue$ per capire se la lista è vuota.

\newpage
\section{Sottosequenze}

\subsection{Sottosequenza singola crescente (LGS)}
Una sottosequenza di una stringa $x\:=\:<x_1,\:x_2,\:\dots,\:x_n>$ è un insieme di indici $i_1,\:i_2,\:\dots\:i_k$ con $k \leq n$ con indici $k$ strettamente crescenti. Gli indici non devono necessariamente essere consecutivi.

Esempio di sottosequenza con una stringa:

$X = <\stackon{C}{1}, \stackon{C}{2}, \stackon{A}{3}, \stackon{B}{4}, \stackon{B}{5}, \stackon{C}{6}, \stackon{A}{7}, \stackon{B}{8}, \stackon{D}{8}>$

$Z = <C, B, B, D>$

$Z$ è sottosequenza di $X$ alle posizioni:
$$(1\ 4\ 5\ 9), (1\ 4\ 8\ 9), (1\ 5\ 8\ 9), (2\ 4\ 5\ 9), (2\ 4\ 8\ 9), (2\ 5\ 8\ 9)$$

\textbf{L'ordine non è casuale}: inizia sempre dalla prima lettera disponibile, poi cerca tutte le alternative a partire dall'ultima posizione. Una volta cambiata la prima lettera, la sequenza viene ricontrollata.

Variabili:
\begin{itemize}
	\item $C[1\dots n]$, numeri interi che indicano le posizioni;
	\item $C[i]$, numero di elementi della più lunga sottosequenza crescente da 1 a $i$ che termina in $x_i$.
\end{itemize}

\textbf{Caso base}: $c[1] = 1,\ max = 0$;

\textbf{Passo ricorsivo}: $c[i] = max\{c[j]\ |\ 1 < j < i \lor x_j < x_i\} + 1\ (se\ stesso)$;

\textbf{Soluzione}: $maxtot = max\{c[i]\ |\ 1 \leq i \leq n\}$.

L'algoritmo iterativo illustrato calcola la lunghezza massima delle sottostringhe in ordine crescente. Il valore $c[i]$ contiene l'ultima posizione della più lunga sottosequenza. \\
Per trovare la sequenza è sufficiente salvare l'elemento compatibile precedente a $c[i]$ per poi fare backtracking. In questo modo la sequenza può essere generata utilizzando $maxtot$. \par 

\begin{algorithm}[H]
	\caption{Ricerca iterativa della sottostringa}
	\begin{algorithmic}
		\Function{longest\_substring}{$x$}
		\State $maxtot \gets 1$
		\State $c[1] \gets 1$
		\For{$i \gets 2$ \textbf{to} $n$ \textbf{step} $1$}
			\State $max \gets 0$
			\For{$j \gets 1$ \textbf{to} $(i - 1)$}
				\If{$x[j] < x[i] \land c[j] > max$}
					\State $max \gets c[j]$
				\EndIf
				\State $c[i] \gets max + 1$
				\If{$c[i] > maxtot$}
					\State $maxtot \gets c[i]$
				\EndIf
			\EndFor
		\EndFor
		\State \Return $maxtot$
		\EndFunction
	\end{algorithmic}
\end{algorithm}

\textbf{Tempo}: $\Theta(n^2)$

\textbf{Spazio}: $\Theta(n)$ (riducendo lo spazio, il tempo aumenta)

\subsection{Longest Common Substring}
\textbf{L}ongest \textbf{C}ommon \textbf{S}ubstring è il problema della sottostringa comune più lunga. Si ha:

$X\:=\:<x_1,\:x_2,\:\dots,\:x_n>$,

$Y\:=\:<y_1,\:y_2,\:\dots,\:y_n>$.

$Z$ è sottosequenza comune a $X$, $Y$ se è sottosequenza di $X$ e contemporaneamente sottosequenza di $Y$. Il problema viene esteso al caso con due stringhe, si ricerca la più lunga sottosequenza con caratteri in ordine (come definito precedentemente).

NB: la funzione non è $1\ :\ 1$,  ci possono essere ``salti'' da una lettera all'altra (altrimenti sarebbe un semplice algoritmo che utilizza un ciclo $for$).

Le soluzioni iterative più banali hanno un tempo di $2^n$, quindi si deve ricorrere alla programmazione dinamica.

Le istanze relative ai sottoproblemi sono $X_i$, $Y_j$, le quali sono $n + 1$ e $m + 1$ prefissi del problema dato. Si deve definire che cosa rappresentano gli indici. Un sottoproblema generico è identificato da una coppia di indici. 

A ogni soluzione è associata una lunghezza: in questo caso la lunghezza della massima sottosequenza comune $K$ è la maggiore tra le lunghezze di tutte le sottosequenze comuni $W$. 

$Z_k = LCS(X_i,\ Y_j)$. \\
Se $x_i = y_j$, $LCS(x_{i-1}, y_{j-1})\ |\ x_i$ (ultimo simbolo). \\
Altrimenti, se $z_k \neq x_i,\ Z_k = LCS(X_i,\ Y_j) = LCS(X_{i-1},\ Y_j)$ e $C_{i, j} = C_{i-1,\ j}$.
Se $z_k \neq y_j,\ Z_k = LCS(X_i,\ Y_j) = LCS(X_i,\ Y_{j-1})$ e $C_{i, j} = C_{i,\ j-1}$.

Si cerca un algoritmo tale che
$$LCS(x,\ y) \rightarrow LCS(x-\{A\},\ y-\{A\})$$

La struttura dati per risolvere il problema in modo dinamico è una \textbf{matrice}, che indica come sono allineate le sequenze restringendo il numero di caratteri. L'algoritmo controlla i primi $i$ elementi di $X$ ed i primi $j$ caratteri di $Y$.

\begin{example}{}{}

\begin{align*}
	X &= <\stackon{A}{1}, \stackon{B}{2}, \stackon{C}{3}, \stackon{B}{4}, \stackon{D}{5}, \stackon{A}{6}, \stackon{B}{7}> \\
	Y &= <\stackon{B}{1}, \stackon{D}{2}, \stackon{C}{3}, \stackon{A}{4}, \stackon{B}{5}, \stackon{A}{6}>
\end{align*}

\begin{center}
	\begin{tabular}{ c | c | c | c | c | c | c | c | }
		X/Y & 0 & B & D & C & A & B & A \\ \hline
		0 & 0 & 0 & 0 & 0 & 0 & 0 & 0 \\ \hline
		A & 0 & 0 & 0 & 0 & 1 & 1 & 1 \\ \hline
		B & 0 & 1 & 1 & 1 & 1 & 2 & 2 \\ \hline
		C & 0 & 1 & 1 & 2 & 2 & 2 & 2 \\ \hline
		B & 0 & 1 & 1 & 2 & 2 & 3 & 3 \\ \hline
		D & 0 & 1 & 2 & 2 & 2 & 3 & 3 \\ \hline
		A & 0 & 1 & 2 & 2 & 3 & 3 & 4 \\ \hline
		B & 0 & 1 & 2 & 2 & 3 & 4 & 4 \\ \hline
	\end{tabular}
\end{center}

\end{example}

La prima riga e la prima colonna sono fissi 0, e il controllo inizia dalla prima riga di 0. Non appena l'algoritmo trova un match, copia nella casella della matrice corrispondente il valore contenuto nella casella precedente sulla diagonale a sinistra, incrementandolo di 1.

Se i due caratteri confrontati sono diversi, viene copiato nella casella il valore massimo tra quello a sinistra e quello sopra, e l'algoritmo prosegue.

La casella alla posizione $(n, m)$ rappresenta la massima lunghezza. 

\textbf{Variabile}:
$c[n, m]$ è una matrice il cui elemento $C[i, j]$ contiene la lunghezza della stringa più lunga fra gli i caratteri di $X$ ed i $j$ caratteri di $Y$.

\textbf{Caso base}:
$$C[i, j] = 0\ se\ i = 0\ \lor j = 0 $$
\textbf{Caso generico}:
$$c[i, j] = \begin{cases}
	C[i-1, j-1] + 1 & se\ x_i = y_j \\
	max\{C[i-1, j], C[i, j-1]\} & se\ x_i \neq y_j
\end{cases}$$

\begin{algorithm}
	\caption{LCS}
	\begin{algorithmic}
		\Function{LCS}{$X, Y$}
			\For{$i \gets 0$ \textbf{to} $n$ \textbf{step} $1$}
				\State $C[i, 0] \gets 0$
			\EndFor
			\For{$j \gets 0$ \textbf{to} $m$ \textbf{step} $1$}
				\State $C[0, j] \gets 0$
			\EndFor
			\For{$i \gets 1$ \textbf{to} $n$ \textbf{step} $1$}
				\For{$j \gets 1$ \textbf{to} $m$ \textbf{step} $1$}
					\If {$X[i] == Y[j]$}
						\State $C[i, j] = C[i-1, j-1] + 1$
					\Else
						\State $C[i, j] = \Call {max}{C[i-1, j], C[i, j-1]+1}$
					\EndIf
				\EndFor
			\EndFor
			\State \Return $C[n, m]$
		\EndFunction
	\end{algorithmic}
\end{algorithm}

Il problema può essere esteso con la richiesta della sottostringa più lunga crescente. In questo caso è necessario controllare la compatibilità di ogni lettera con le lettere precedenti.
Per ottenere la sottostringa, oltre alla lunghezza, bisogna salvare in $S[i, j]$ (una matrice aggiuntiva) i movimenti eseguiti da $S[n, m]$ a ritroso per ricomporre la stringa.

Il tempo di computazione è limitato superiormente dai due cicli $for$ dell'algoritmo, quindi $\theta(nm)$ che è approssimabile a $\theta(n^2)$

Lo spazio richiesto è $\theta(nm)$ (la matrice), e può essere ridotto tenendo solo l'ultima riga oppure sostituendo le righe e colonne di $0$ con degli $if$.

\subsection{LCS crescente}
Date due sequenze $X$, $Y$, si vuole trovare la sottostringa comune in ordine crescente più lunga. I sottoproblemi vengono formulati in questo modo: qual'è la più lunga sottostringa che termina con $x_i,\ y_i$? 

Bisogna controllare ogni volta se la sottosequenza comune precedente è compatibile (crescente) con l'ultimo carattere aggiunto, per mantenere l'ordine. 

Si ha che se $X_i = Y_i$ viene calcolata la LCS dei caratteri precedenti + 1; altrimenti l'algoritmo restituisce $\emptyset$.

Variabili:
\begin{itemize}
	\item $X_n$ con $|X_n| = n$, $Y_m$ con $|Y_m| = m$;
	\item $C[n, m]$, una matrice;
	\item $C[i, j]$, il numero di caratteri che compongono la più lunga sottosequenza comune a $X$, $Y$ crescente che termina con ($x_i = y_i$).
\end{itemize}

L'equazione di ricorrenza del problema è
$$\begin{cases}
C[i, j] = 0 & se\ x_i \neq y_i \\
C[i, j] = max\{C[h, k]\ |\ 1 \leq h < i,\ 1 \leq k < j,\ x_h < x_i,\ y_k < y_j\} + 1 & x_i = y_j \\
\end{cases}$$

Questo può essere espresso semplicemente come
$$max\{C[i, j]\ |\ 1 \leq i \leq n,\ 1 \leq j \leq m\}$$

L'algoritmo funziona tramite una matrice similmente a LCS: le caselle corrispondenti a caratteri diversi vengono settate a 0. 

Quando un carattere è uguale viene inserito il valore della più lunga sottosequenza comune compatibile precedente: in altre parole si cerca la casella con il valore più alto tale che l'elemento sia in ordine crescente. \\
Alla fine della ricerca del massimo, il numero viene incrementato di 1. Per eseguire queste operazioni sono necessari 4 cicli $for$.

Al completamento della scansione della matrice e dell'inserimento dei valori, bisogna trovare il massimo assoluto tra di essi: perciò è necessaria un'ulteriore ricerca tramite cicli $for$.

Questo rende l'algoritmo abbastanza pesante computazionalmente: il tempo è di circa $\theta(n^2m^2)$, approssimabile a $\theta(n^4)$.

\subsubsection{Algoritmo}

\begin{algorithm}[H]
	\caption{LCS\_crescente}
	\begin{algorithmic}
		\Function{LCS\_crescente}{$X, Y$}
			\For{$i \gets 1$ \textbf{to} $n$ \textbf{step} $1$}
				\For{$j \gets 1$ \textbf{to} $m$ \textbf{step} $1$}
					\If {$X[i] \neq Y[j]$}
						\State $C[i, j] \gets 0$
					\Else
						\State $max \gets 0$
						\For{$h \gets 1$ \textbf{to} $i-1$ \textbf{step} $1$}
							\For{$k \gets 1$ \textbf{to} $j-1$ \textbf{step} $1$}
								\If {$C[h, k] > max \land x[k] < x[i]}$
									\State $max \gets C[h, k]$
								\EndIf
							\EndFor
						\EndFor
						\State $C[i, j] \gets max + 1$
					\EndIf
				\EndFor
			\EndFor
			\State $maxtot \gets 0$
			\For{$i \gets 1$ \textbf{to} $n$ \textbf{step} $1$}
				\For{$j \gets 1$ \textbf{to} $m$ \textbf{step} $1$}
					\If {$C[i, j] > maxtot$}
						\State $maxtot \gets C[i, j]$
					\EndIf
				\EndFor
			\EndFor
			\State \Return $maxtot$
		\EndFunction
	\end{algorithmic}
\end{algorithm}

\subsubsection{Implementazione}
\lstinputlisting[language=Python]{Algoritmi/lcs_crescente.py}


\subsection{Heaviest Common Subsequence}
Questa è una variante di LCS, che associa a ogni elemento $X_i$ o $Y_j$ un peso. L'obiettivo è trovare la sequenza comune di peso maggiore (non necessariamente la più lunga).

A ogni aggiornamento della sequenza comune, va aggiunto il peso del nuovo elemento; il resto dell'algoritmo è concettualmente uguale. La programmazione dinamica aiuta a ottenere la soluzione ottimale.





\newpage
\section{Weighted Interval Scheduling}

\begin{center}
\begin{ganttchart}[vgrid,hgrid]{0}{12}
    \gantttitlelist{0,...,12}{1} \\
    \ganttbar[bar/.append style={fill=arancione}]{a}{0}{6} \\
    \ganttbar[bar/.append style={fill=verde}]{b}{1}{4} \\
    \ganttbar[bar/.append style={fill=azzurro}]{c}{3}{5} \\
    \ganttbar[bar/.append style={fill=arancione}]{d}{3}{8} \\
    \ganttbar[bar/.append style={fill=verde}]{e}{4}{7} \\
    \ganttbar[bar/.append style={fill=azzurro}]{f}{5}{9} \\
    \ganttbar[bar/.append style={fill=arancione}]{g}{6}{10} \\
    \ganttbar[bar/.append style={fill=verde}]{h}{8}{11}
\end{ganttchart}
\end{center}

\subsection{Il problema}
Ci sono $n$ attività, con $n \in \mathbb{N},\ n \geq 1$. Ogni attività (job) $i \in \{1,\ \dots,\ n\}$ ha le seguenti proprietà:
    \begin{enumerate}
        \item Un job $i$ inizia ad $s_i$, finisce a $f_i$, ed ha un peso (o valore) $v_j$;
        \item Due jobs sono \textbf{compatibili} se non si sovrappongono: $[s_i, f_i) \cap [s_j, f_j) = \emptyset$.
    \end{enumerate}

Scopo: trovare il massimo subset di \textbf{peso} di jobs mutualmente compatibili, $A \subseteq \{1,\ \dots,\ n\}$ per cui $\forall i,\ j \in A\ con\ i \neq j\, [s_i, f_i) \cap [s_j, f_j) = \emptyset$.

Ciò significa semplicemente: trovare la sequenza di jobs mutualmente compatibili (cioè una lista di jobs che non si sovrappongono), la cui \textbf{somma combinata} del peso (o valore) è la massima possibile. 
La compatibilità viene espressa tramite la seguente formula: 
$$comp : \mathcal{P}(\{1, \ \dots, \ n\}) \rightarrow \{true,\ false\}\ \forall \ A \subseteq \{1, \ \dots, \ n\}$$

In altre parole:
$$comp(A) = 
\begin{cases}
	true & se\ A\ contiene\ attivit\grave{a}\ mutualmente\ compatibili \\
	false & altrimenti
	\end{cases}$$	
	
Definendo la funzione in base al peso $v$, si ha
$$v : \mathcal{P}(\{1, \ \dots, \ n\}) \rightarrow \R\ \forall \ A\ \in\ \mathcal{P}(\{1, \ \dots, \ n\}),\ A \subseteq \{1, \ \dots, \ n\}$$
$$v(A) = 
\begin{cases}
\sum_{i \in A}& se\ A \neq \emptyset \\
0 & se\ A = \emptyset
\end{cases}$$	

Si ricorda che usare un algoritmo di tipo \textbf{greedy} non è la scelta più saggia, poichè i jobs hanno tutti peso (o valore) differenti. \\
Esempio: 
\begin{center}
\begin{ganttchart}[vgrid,hgrid]{0}{12}
    \gantttitlelist{0,...,12}{1} \\
    \ganttbar[bar/.append style={fill=arancione}]{peso = 999}{0}{10} \\
    \ganttbar[bar/.append style={fill=verde}]{peso = 1}{0}{3}
\end{ganttchart}
\end{center}

\subsection{Approccio}
Il problema consiste nel trovare un sottoinsieme di $\{1,\ \dots,\ n\}$ composto di attività mutualmente compatibili e di valore massimo. \\
Quindi, un'istanza $n \in \mathbb{N}$ con $n \geq 1$ tale che $\forall i \in \{1,\ \dots,\ n\}$ si ha $s_i$, $f_i$, $v_i$ massimi.

La soluzione più ovvia sarebbe il brute-force, che ha una complessità di $2^n$. 
La programmazione dinamica risolve il problema dividendolo in sottoproblemi, e applicando $comp$ ai sottoinsiemi fino ad arrivare al caso base (insieme vuoto). A ogni step si riduce l'insieme delle posizioni di 1.

Si ha che, se l'insieme ha una sola attività, $s_1 = \{1\}$ (caso base). \\
Se l'insieme è vuoto, $s_0 = \{\emptyset\}$ (caso base). \\
Si suppone di aver già risolto i problemi $\{0,\ 1,\ \dots,\ i-1\}$ per poi risolvere il sottoproblema $i$ (ci sono $n + 1$ sottoproblemi).

La soluzione dell'algoritmo è $S \subseteq \{1,\ \dots,\ n\}$ tale che:
\begin{enumerate}
\item $comp(S) = true$;
\item $v(S) = max\{v(A)\}$ con $\subseteq \{1,\ \dots,\ n\},\ comp(A) = true$.
\end{enumerate}

Si potrebbe ordinare i jobs secondo il loro tempo di fine: \\
$f_1 \leq f_2 \leq \dots \leq f_n$. \\
$\forall i \in (\{1, \ \dots, \ n\}),\ p(i) = max\{j | j < i, j\ \grave{e}\ compatibile\ con\ i\}$.

\begin{minipage}{\textwidth}
    \begin{minipage}{0.75 \textwidth}
        \begin{ganttchart}[vgrid,hgrid]{0}{12}
            \gantttitlelist{0,...,12}{1} \\
            \ganttbar[inline,bar/.append style={fill=arancione}]{1}{1}{3} \\
            \ganttbar[inline,bar/.append style={fill=verde}]{2}{3}{4} \\
            \ganttbar[inline,bar/.append style={fill=azzurro}]{3}{0}{5} \\
            \ganttbar[inline,bar/.append style={fill=arancione}]{4}{4}{6} \\
            \ganttbar[inline,bar/.append style={fill=verde}]{5}{3}{7} \\
            \ganttbar[inline,bar/.append style={fill=azzurro}]{6}{5}{8} \\
            \ganttbar[inline,bar/.append style={fill=arancione}]{7}{6}{9} \\
            \ganttbar[inline,bar/.append style={fill=verde}]{8}{8}{10}
        \end{ganttchart}
    \end{minipage}
    \begin{minipage}{0.2 \textwidth}
        \begin{table}[H]
        \begin{tabular}{|l|l|}
        \hline
        j & $p(j)$ \\ \hline
        0 & -      \\ \hline
        1 & 0      \\ \hline
        2 & 0      \\ \hline
        3 & 0      \\ \hline
        4 & 1      \\ \hline
        5 & 0      \\ \hline
        6 & 2      \\ \hline
        7 & 3      \\ \hline
        8 & 5      \\ \hline
        \end{tabular}
        \end{table}
    \end{minipage}
\end{minipage}


Ci sono due possibilità:
\begin{enumerate}
	\item $i \in S_i$, allora $S_i = S_{p(i)} \cup \{i\}$ e $v(S_i) = v(S_{p(i)}) + v_i$;
	\item $i \notin S_i$, allora la soluzione coincide con il sottoproblema precedente e $v(S_i) = v(S_{i-1})$.
	\end{enumerate}
La condizione del controllo è
$$v(S_i) = max\{v(S_{i-1}),\:v(S_{p(i)}) + 1\}$$
La tabella rappresenta con $p(i)$ il primo indice a ritroso di job compatibile con l'attività $i$. 

\subsection{Algoritmo}

\begin{algorithm}[H]
	\caption{Weighted Interval Scheduling}
	\begin{algorithmic}
		\Function{WIS-PD}{$n$}
			\State $M[0] \gets 0$
			\State $M[1] \gets v_1$
			\For{$i \gets 2$ \textbf{to} $n$ \textbf{step} $1$}
				\If{$M[p_i] + v_1 > M[i-1]$}
					\State $M[i] \gets M[p(i)] + v_i$
				\Else
					\State $M[i] \gets M[i-1]$
				\EndIf
			\EndFor
		\EndFunction
	\end{algorithmic}
\end{algorithm}

Le soluzioni dei sottoproblemi vegono memorizzate in un array $M$ per evitare il tempo esponenziale di un algoritmo ricorsivo.
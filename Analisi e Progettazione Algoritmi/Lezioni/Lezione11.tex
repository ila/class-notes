\section{DFS}
DFS è un algoritmo che visita un grafo in profondità, al contrario di BFS che visita subito tutti i vertici adiacenti. La struttura dati utilizzata è la pila: ogni elemento viene subito rimosso. 

Un'altra differenza è che DFS permette di ottenere informazioni sui nodi di un grafo come insieme, per esempio il numero di componenti connesse e le loro caratteristiche, mentre BFS è più utile per il cammino minimo o per i singoli vertici.

Similmente, invece, DFS ricorre all'utilizzo dei colori: bianco, grigio e nero, con gli stessi significati di BFS. Se il grafo non è orientato, non è detto che tutti i nodi vengano scoperti; per questo motivo la visita viene lanciata ogni volta che viene trovato un elemento bianco.

Viene settato un clock $d$ per ogni vertice, che indica a quale unità di tempo esso è stato scoperto. Analogamente, $f$ è il tempo relativo alla fine della visita di quel vertice. 

Per capire se gli archi sono in avanti oppure semplicemente di attraversamento (portano a nodi neri), è sufficiente guardare il tempo di fine. Se i segmenti dei tempi sono disgiunti, è un arco di cross: se uno è sottoinsieme dell'altro, sono disgiunti.

La ricorsione è utile per simulare una pila senza doverla gestire direttamente: le chiamate ricorsive vengono eseguite dallo stack. Se necessario, l'algoritmo ricorsivo viene lanciato su ogni nodo.

\subsection{Algoritmo}
DFS prende in ingresso un grafo $G$ e per ogni vertice inizializza il colore, il predecessore, il tempo di scoperta e di fine visita (implicitamente).

Se il vertice è bianco, chiama l'algoritmo \textit{DFS\_visit} che esegue la visita in profondità vera e propria.

Una volta scoperto il vertice, il tempo di esecuzione viene incrementato e salvato, e il colore diventa grigio. La procedura deve analizzare tutti i nodi sottostanti e, una volta finito, segnare il tempo di fine. 

Tra l'inizio e la fine della scoperta, per ogni vertice della lista di adiacenza viene ricontrollato il colore. Se esso è bianco, viene visitato ricorsivamente il sottografo richiamando \textit{DFS\_visit}. 

% algoritmo DFS
\begin{algorithm}[H]
	\caption{DFS}
	\begin{algorithmic}
		\Function{DFS}{G}
			\For{$v \in G$}
				\State $c[v] \gets $ White
				\State $p[v] \gets $ nil
			\EndFor
			\State $time \gets 0$
			\For{$v \in G$}
				\If{$c[v] == $ White}
					\State \Call{DFS\_visit}{G,v}
				\EndIf
			\EndFor
		\EndFunction
		\Function{DFS\_visit}{G,v}
			\State $time \gets time + 1$
			\State $d[v] \gets time$
			\State $c[v] \gets $ Gray
			\For{$a \in Adj[v]$}
				\If{$c[a] == $ White}
					\State \Call{DFS\_visit}{G,a}
				\EndIf
			\EndFor
			\State $time \gets time + 1$
			\State $f[v] \gets time$
			\State $c[v] = $ Black
		\EndFunction
	\end{algorithmic}
\end{algorithm}
% DFS_visit

Il caso migliore per l'algoritmo è chiaramente il tempo costante. Il caso peggiore si verifica quando tutti i nodi vengono visitati con \textit{DFS\_visit}. \\
Tempo medio: $\Theta(\abs{V}+ \abs{E})$.

\subsection{Proprietà}
Teorema delle parentesi: dati due intervalli $i_1 = (d[u], f[u])$ e $i_2 = (d[v], f[v])$, confrontandoli si possono avere tre casi:
\begin{enumerate}
	\item $i_1 \cap i_2 = \emptyset$, nodi disgiunti;
	\item $i_1 \subseteq i_2$, figlio;
	\item $i_1 \supseteq i_2$, padre.
\end{enumerate}
Non esistono casi in cui gli intervalli siano accavallati.

Un lato si può etichettare in questi modi:
\begin{itemize}
	\item Tree-edge, albero di scoperta (senza cicli);
	\item Back-edge (lato all'indietro), arco che porta a un vertice grigio;
	\item Forward-edge (lato in avanti), porta a un vertice nero, $d[u] < f[v]$;
	\item Cross-edge, lato di attraversamento da un grafo all'altro, $d[u] > f[v]$. 
\end{itemize}

In un grafo non orientato, è possibile avere tree-edge e back-edge, ma non sono mai presenti forward-edge perché ogni arco viene prima esplorato dal lato opposto. 

\subsubsection{Ordinamento topologico}
L'ordinamento topologico è utile per stabilire un ordine logico su grafi particolari, che rientrano nella categoria dei DAG (grafi diretti senza cicli). 

Si fissa un nodo che deve per forza essere il primo, e poi ci sono delle possibili scelte (che rispettino i vincoli). 

Viene lanciato DFS sul grafo che permette di visitare i nodi nell'ordine richiesto, mettendoli in ordine man mano.

\begin{example}{}{}
	Modificare DFS affinché ad ogni lato del grafo venga messa l'etichetta corretta e stampata a video

	\begin{algorithm}[H]
		\caption{DFS}
		\begin{algorithmic}
			\Function{DFS\_visit}{G,v}
				\State $time \gets time + 1$
				\State $d[v] \gets time$
				\State $c[v] \gets $ Gray
				\For{$a \in Adj[v] \backslash P[v] $} \Comment{Si rimuove $P[v]$ solo se il grafo non è orientato}
					\If{$c[a] == $ White}
						\State \Call{Print}{``(v,a), Tree''}
						\State \Call{DFS\_visit}{G,a}
					\ElsIf{$c[a] == $ Gray}
						\State \Call{Print}{``(v,a), Back''}
					\ElsIf{$c[a] == $ Black} \Comment{Solo se il grafo è orientato}
						\If{$d[a] < f[a]$}
							\State \Call{Print}{``Forward''}
						\Else
							\State \Call{Print}{``Cross''}
						\EndIf
					\EndIf
				\EndFor
			\EndFunction
		\end{algorithmic}
	\end{algorithm}
\end{example}
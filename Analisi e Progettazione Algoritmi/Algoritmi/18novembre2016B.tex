\documentclass{article}

\input{../../packages.tex}
\input{../../commons.tex}

\begin{document}
\section{I compitino, 18 novembre 2016 B}
Si richiede un algoritmo che sfrutta la tecnica della programmazione dinamica per risolvere il seguente problema: si considerino due sequenze $s_1$ e $s_2$ di lettere dell'alfabeto italiano, di lunghezza $n$ e $m$ rispettivamente. 

A ogni lettera dell'alfabeto è associato un numero naturale, mediante una funzione $f$ prefissata.

Si vuole calcolare la lunghezza di una più lunga sottosequenza comune a $s_1$ e $s_2$ nella quale i caratteri appaiono in ordine lessicografico crescente e i numeri ad essi associati appaiono in ordine decrescente.

\subsection{Variabili che servono per risolvere il problema}
La programmazione dinamica serve per risolvere un problema considerando di aver già risolto tutti i sottoproblemi (prefissi), e potendo usare la soluzione di essi per calcolare la soluzione del problema.

Per immagazzinare le soluzioni dei sottoproblemi viene utilizzata una matrice $c$, di dimensione $n \times m$. A ogni sottoproblema $i, j$ corrisponde una casella $c[i, j]$ che contiene il valore di una delle sottosequenze comuni più lunghe che rispettano i vincoli dell'algoritmo.

Con $s_1$ di lunghezza $n$ e $s_2$ di lunghezza $m$, il problema finale avrà dimensione $n,\ m$. Si definiscono indici $i, j$ che indicano ogni sottoproblema, tale che $0 \leq i \leq n$ e $0 \leq j \leq m$. In totale, considerando anche i casi limite con prefisso che corrisponde alla stringa vuota, si hanno $(n + 1)(m + 1)$ sottoproblemi.

Le sottostringhe $s_{1, i}$ e $s_{2, j}$ sono prefissi di lunghezza rispettivamente $i$ e $j$ delle stringhe di partenza $s_1$ e $s_2$ di lunghezza $n$ e $m$, con $0 \leq i \leq n$ e $0 \leq j \leq m$.

In questo caso, il problema è definito come: \\
\textit{trovare la lunghezza di una delle più lunghe sottosequenze di $s_1$, $s_2$ tale che i caratteri siano in ordine lessicografico crescente e i numeri associati in ordine decrescente}.

Ogni sottoproblema è definito come: \\
\textit{trovare la lunghezza di una delle più lunghe sottosequenze di $s_{1, i}$, $s_{2, j}$, con $0 \leq i \leq n$ e $0 \leq j \leq m$, tale che i caratteri siano in ordine lessicografico crescente e i numeri associati in ordine decrescente}.
	
Per risolvere questo problema, nel caso in cui $s_{1, i} = s_{2, j}$ è necessario conoscere il valore numerico dell'ultimo carattere della più lunga sottosequenza comune precedente, in modo da poterlo confrontare con il valore corrente.

Il problema viene esteso: \\
\textit{trovare la lunghezza di una delle più lunghe sottosequenze di $s_{1, i}$, $s_{2, j}$ tale che i caratteri siano in ordine lessicografico crescente e i numeri associati in ordine decrescente \textbf{e che termini con $s_{1, i}$, $s_{2, j}$} (nel caso in cui esse coincidano)}.

La variabile associata a ogni problema è: \\
\textit{$c[i, j]$ che contiene la lunghezza di una delle più lunghe sottosequenze di $s_{1, i}$, $s_{2, j}$ che abbia lettere crescenti e numeri associati decrescenti (nel caso in cui $s_{1, i}, s_{2, j}$ coincidano), e contiene 0 altrimenti}.

Si ricorda che ognuna di queste variabili è considerabile come una \textbf{black-box}: si può utilizzare ma non è possibile conoscerne il contenuto (si assume di aver risolto i sottoproblemi di dimensione minore).
	
\subsection{Equazione di ricorrenza}
Un algoritmo ricorsivo non è efficace, perché calcolerebbe più volte lo stesso risultato, quindi si ricorre alla programmazione dinamica per risolvere il problema.

Le soluzioni dei sottoproblemi non sono ancora note, ma è possibile utilizzarle per calcolare le soluzioni successive. L'unico caso conosciuto è il caso limite: se una delle due sequenze è vuota, la lunghezza della più lunga sottosequenza è 0 ($i = 0 \lor j = 0$).

Questo si può esprimere nel seguente modo:
$$c[i, 0] = 0\ con\ \{0 \leq i \leq n\},\ c[0, j] = 0\ con\ \{0 \leq j \leq m\}$$

Si introduce il sottoproblema tale per cui $s_{1, i} \neq s_{2, j}$: in questo caso, non è possibile trovare la più lunga sottosequenza comune che termini con $s_{1, i}, s_{2, j}$ con valore maggiore di 0. Di conseguenza si assegna 0 alla casella corrispondente.
$$c[i, j] = 0 \text{ con } s_{1, i} \neq s_{2, j}$$

Avendo risolto il caso con il prefisso vuoto, i sottoproblemi da risolvere diventano al più $mn$. L'equazione di ricorrenza per un generico sottoproblema $i, j$ è:
$$c[i, j] = \begin{cases}
1 + max\{c[h, k]\}\ \text{con}\ \{1 \leq h < i,\ 1 \leq k < j\} & \text{ se } s_{1, h} < s_{1, i},\ f(s_{1, i}) < f(s_{1, h}) \\
1 & \text{ altrimenti}  
\end{cases}$$
	
\subsection{Soluzione del problema}
La più lunga sottosequenza comune crescente con numeri decrescenti tra $s_{1, i}$ e $s_{2, j}$ è data dal massimo valore tra le soluzioni delle più lunghe sottosequenze comuni per i sottoproblemi di dimensione minori.

La soluzione del problema corrisponde al massimo tra tutte le caselle della matrice.
$$max\{c[i, j]\ |\ 0 \leq i \leq n \land 0 \leq j \leq m\}$$

L'algoritmo funziona in questo modo: in ogni casella di $c$ viene inserito 0 se i due caratteri in posizione $i, j$ sono diversi, altrimenti viene cercato il massimo nella sottomatrice precedente utilizzando due indici ausiliari $h, k$. Il valore corrispondente deve rispettare le seguenti condizioni: il numero precedente dev'essere maggiore, e la lettera precedente dev'essere minore della lettera associata.

Il risultato ottenuto viene incrementato di 1, perché è stato trovato un altro carattere uguale.

\subsection{Algoritmo in pseudocodice}
Viene illustrata la soluzione bottom-up mediante un algoritmo iterativo, risolvendo ogni sottoproblema una volta sola.
\begin{algorithm}[H]
	\caption{LGCS con numeri decrescenti}
	\begin{algorithmic}
		\Function{LGCS numeri decrescenti}{$s_1,\ s_2$}
			\For {$i \gets 0 \textbf{ to } n$}
				\State $c[i, 0] \gets 0$
			\EndFor
			\For {$j \gets 0 \textbf{ to } m$}
				\State $c[0, j] \gets 0$
			\EndFor
			\State $max \gets 0$
			\For {$i \gets 1 \textbf{ to } n$}
				\For {$i \gets 1 \textbf{ to } n$}
					\If {$s_1[i] \neq s_2[j]$}
						\State $c[i, j] \gets 0$
					\Else
						\State $temp \gets 0$
						\For {$h \gets 1 \textbf{ to } i-1$}
							\For {$k \gets 1 \textbf{ to } j-1$}
								\If {$s_{1, i} > s_{1, h} \land f(s_{1, i}) < f(s_{1, h}) \land c[h, k] > temp$}
									\State $temp \gets c[h, k]$
								\EndIf
							\EndFor
						\EndFor
						\State $c[i, j] \gets 1 + temp$
					\EndIf
					\If {$c[i, j] > max$}
						\State $max = c[i, j]$
					\EndIf
				\EndFor
			\EndFor
		\EndFunction
	\end{algorithmic}
\end{algorithm}

\subsection{Valutazione del tempo di esecuzione}
Il tempo di esecuzione di questo algoritmo è approssimativamente $O(n^2m^2)$. Ciò deriva dai cicli \textit{for} innestati: la matrice viene interamente analizzata casella per casella, e poi c'è la ricerca del massimo valore nella sottomatrice.

Lo spazio utilizzato è $\Theta(nm)$, per la matrice.

\end{document}

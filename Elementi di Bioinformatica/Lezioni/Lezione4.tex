\newpage
\section{Comandi Unix}
\begin{multicols}{2}
\subsection{Essenziali}
\begin{itemize}
	\item \texttt{ls} vedere il contenuto di una directory;
	\item \texttt{cd} cambiare la directory corrente;
	\item \texttt{pwd} vedere la directory corrente;
	\item \texttt{cp} copiare un file;
	\item \texttt{mv} spostare o rinominare un file (o una directory);
	\item \texttt{rm} rimuovere un file;
	\item \texttt{cat} mostrare il contenuto di un file;
	\item \texttt{mkdir} creare un directory;
	\item \texttt{rmdir} eliminare una directory vuota;
	\item \texttt{man} mostrare la pagina di manuale di un comando.
\end{itemize}

\subsection{Importanti}
\begin{itemize}
	\item \texttt{gzip}, \texttt{bzip2}, \texttt{xz} comprimere un file;
	\item \texttt{tar} creare, verificare, ripristinare un archivio;
	\item \texttt{grep} cercare un pattern in un testo;
	\item \texttt{head} mostrare le prime righe di un file;
	\item \texttt{tail} mostrare le ultime righe di un file.
\end{itemize}

\subsection{Altri concetti}
\begin{itemize}
	\item \texttt{stdin} standard input, normalmente la tastiera;
	\item \texttt{stdout} standard output, normalmente lo schermo;
	\item \texttt{stderr} standard error,normalmente lo schermo;
	\item \texttt{.} standard input/output/error, possono essere un file;
	\item \texttt{>} redirezione di standard output (sovrascrive);
	\item \texttt{>>} redirezione di standard output (appende);
	\item \texttt{<} redirezione di standard input;
	\item \texttt{2>} redirezione di standard error;
	\item \texttt{|} pipe, lo stdout del primo programma diventa stdin del secondo.
\end{itemize}

\subsection{Altri comandi}
\begin{itemize}
	\item \texttt{ssh} per collegarsi da un altro computer/server;
	\item \texttt{scp} per copiare un file da/a un altro computer;
	\item \texttt{rsync} come scp, ma copia solo se necessario;
	\item \texttt{make} per compilare solo quello che serve;
	\item \texttt{tr} cambiare alcuni caratteri in altri;
	\item \texttt{sort} ordinare un insieme di righe;
	\item \texttt{uniq} fondere le righe consecutive identiche;
	\item \texttt{cut} estrarre alcune colonne;
	\item \texttt{less} paginatore;
	\item \texttt{find} cercare file.
\end{itemize}

\end{multicols}
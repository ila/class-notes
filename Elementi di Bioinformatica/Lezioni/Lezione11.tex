\section{Allineamento multiplo}
Il problema di allineamento viene esteso, con $k$ stringhe in input. Le sequenze quindi sono un insieme $\{s_1,\ s_2,\ \dots,\ s_k\}$.

\begin{example}{}{}
\begin{tabular}{l *{8}{c}}
	$s_1 = $ & A & C & T & G & G & A & T & ~ \\
	$s_2 = $ & T & G & G & A & T & ~ & ~ & ~ \\
	$s_3 = $ & A & C & G & G & G & A & T & T
\end{tabular}

Un possibile allineamento è:

\begin{tabular}{l *{9}{c}}
	$s_1 = $ & A & - & C & T & G & G & - & A & T \\
	$s_2 = $ & - & - & T & G & G & - & A & - & T \\
	$s_3 = $ & A & C & G & G & - & G & A & T & T
\end{tabular}

In questo caso, è necessario considerare la presenza di più di un indel sulla stessa colonna: è accettabile? Sicuramente non potranno esserci colonne formate unicamente da indel, ma è possibile aggiungerne fino a $k - 1$.
\end{example}

La soluzione ammissibile di questo problema è l'allineamento ottimo che rispetta le seguenti proprietà:
\begin{enumerate}
	\item Le sequenze estese devono avere tutte la stessa lunghezza;
	\item Non devono esserci colonne in cui sono presenti tutti indel.
\end{enumerate}

La matrice dev'essere bidimensionale, per poterla confrontare con PAM e BLOSUM. Per questo motivo si introduce la funzione SP, Sum of Pairs: tutte le coppie di righe sono considerate come sequenze da allineare, ed esse vengono sommate. 

Si usa la somma delle coppie di tutte le sequenze per ottenere uno score complessivo. \\
Si ha $\{s_1,\ s_2,\ \dots,\ s_k\} \rightarrow \{s^{*}_1,\ s^{*}_2,\ \dots,\ s^{*}_k\}$ allineamento ottimo, con valore $\{s^{*}_1[h],\ s^{*}_2[h],\ \dots,\ s^{*}_k[h]\}$. \\
$\sum_{i<j}d(s^*_1[i],\ s^*_k[j])$.

Il problema di ciò è ancora l'inserimento di indel, è permesso averne due finché non sono $k$ ma l'allineamento globale di due sequenze questo non lo prevede.

L'algoritmo sfrutta la programmazione dinamica: viene individuato l'ultimo componente della soluzione ottimale, e per ogni caso sfruttate le soluzioni dei sottoproblemi. L'ultima colonna può trovarsi in $2^k - 1$ casi: ogni sottoinsieme delle stringhe in ingresso potrebbe avere un indel. 

Se $k$ è arbitrario, l'algoritmo è NP-completo. Altrimenti, il tempo computazionale è $O(n^k)$ (dove $n$ è la lunghezza maggiore), con $n^k$ caselle nella matrice, il che è molto elevato. 
	

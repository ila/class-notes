\section{Fibonacci}

Esempi in Python di implementazioni dell'algoritmo di Fibonacci.
\subsection{Soluzione ricorsiva}

\begin{lstlisting}[language=Python]
import argparse

parser = argparse.ArgumentParser(description='Fibonacci.')
parser.add_argument('limit', metavar='N', type=int, nargs=1, help='How many Fibonacci numbers to compute')
args = parser.parse_args()

def fib(n):
    if n <= 1:
        return 1
    else:
        return fib(n-1) + fib(n-2)

print(fib(args.limit[0]-1))
\end{lstlisting}

\subsection{Soluzione iterativa}

\begin{lstlisting}[language=Python]
import argparse

parser = argparse.ArgumentParser(description='Fibonacci.')
parser.add_argument('limit', metavar='N', type=int, nargs=1, help='How many Fibonacci numbers to compute')
args = parser.parse_args()

def fib():
    a,b = 1, 1
    while True:
        yield a
        a, b = b, a+b

for index, f in enumerate(fib()):
    print(f)
    if index == args.limit[0]-1:
        break
\end{lstlisting}
Questa variante utilizza un iteratore, yeld ritorna il valore di $a$ senza uscire dalla funzione. Potenzialmente vengono generati un numero infinito di valori. $enumerate$ restituisce le coppie indice-valore da una lista.
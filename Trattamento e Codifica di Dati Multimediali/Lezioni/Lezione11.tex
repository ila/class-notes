\section{Audio}
Un segnale sonoro è la variazione di pressione sul timpano, percepito nel tempo attraverso l'aria (mezzo di propagazione). Il suono è un'onda di pressione come la luce, ma macroscopico.

L'aumento della frequenza non è linearmente in rapporto con l'ampiezza, ma si comporta seguendo una scala logaritmica. Il range di suoni in grado di essere percepite dagli umani è minore dell'insieme delle frequenze possibili, oltre esso ci sono ultrasuoni e infrasuoni. 

Le curve isofone rappresentano i suoni all'interno di una soglia di udibilità, che può essere superata fino al fastidio. 

Il segnale vocale può avere componenti fino a 10 KhZ, ma in genere se ne utilizzano solo 8, quindi questa è la frequenza di campionamento del segnale telefonico. 
\section{Colore e visione}
L'immagine a colore viene percepita dal sistema visivo grazie alle onde elettromagnetiche, caratterizzate dalla lunghezza d'onda (proporzionale all'inverso della frequenza) e dall'ampiezza. La lughezza d'onda è rapportata il periodo: indica quanto ci mette un ciclo a compiersi.

Lo spettro è in buona parte invisibile, con estremi che corrispondono alle radiazioni ultraviolette e infrarossi. La porzione visibile è misurata in nanometri, con ordini di grandezza molto alti.

Il colore è processato nel cervello da coni e bastoncelli, in numerose tipologie con risposte differenti a seconda della lunghezza d'onda. I bastoncelli sono in percentuale estremamente maggiore, ed essendo essi atti a misurare la variazione d'intensità, gli esseri umani hanno più sensibilità in condizioni di scarsa luce. Le due dimensioni sono separabili. 

L'occhio è più sensibile alle variazioni di luce nel centro dello spettro visibile, quindi lo spettro avrà la forma di una curva e la funzione varia al variare delle lunghezze d'onda. A seconda dei nm della radiazione, alcuni coni saranno attivati, eventualmente fondendo le risposte. 

\subsection{Segnale reale}
L'oggetto assorbe una parte della radiazione e ne ritrasmette un'altra parte (parametri legati alla capacità di riflettere), i cui segnali interferiscono con i recettori umani. Ogni colore può essere descritto da tre parametri (RGB).

I numeri in output si definiscono valori di stimolo, che poi si traducono nei canali RGB. Le distribuzioni continue in funzione di $\lambda$ si trasformano in valori discreti in spazio tridimensionale grazie all'integrale.

Le immagine acquisite da una fotocamera digitale sono differenti da quelle elaborate dall'occhio umano: ogni camera ha una matrice che processa il colore in modo da renderlo simile all'output del cervello, e i valori ottenuti dai sensori sono interpolati. 

I dati sono fatti passare in filtri colore, disposti secondo Bayer pattern. Il verde è il colore recepito meglio, e permette di misurare maggiormente le variazioni di intensità (e conseguentemente i dettagli dell'immagine). 

La somma dei contributi è chiamata sintesi additiva; esiste anche la sintesi sottrattiva, che fa differenze in base allo spettro e ai colori complementari di RGB (ciano, magenta, giallo). 

Esistono rappresentazioni che separano i canali in base alla loro intensità, mettendola in evidenza in base alla frequenza delle onde. 

Le immagini hanno due tipologie di formato: \\
Raster, scalate in base a un numero prefissato di pixel in una griglia di elementi; \\
Vettoriale, rappresentate con formule matematiche.
   



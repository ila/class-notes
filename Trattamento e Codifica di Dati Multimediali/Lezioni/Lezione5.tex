\section{Convoluzione}
La convoluzione è un operatore matematico che descrive l'interazione tra il segnale e il filtro, e ne osserva il comportamento durante il movimento nello spazio. Si definisce nel dominio continuo fra due funzioni $g(x)$ e $f(x)$ tale che:
\begin{enumerate}
	\item L'asse di rappresentazione di uno dei due assi è invertita, $g(t) \rightarrow g(-t)$;
	\item Il segnale invertito viene fatto traslare tra $\infty$ e $-\infty$;
	\item Per ogni traslazione si calcola il prodotto tra il segnale traslato e l'altro non traslato;
	\item Si calcola l'area del prodotto, cioè la somma degli infiniti prodotti composti da una funzione che scorre e una statica.
\end{enumerate}
La convoluzione è l'operatore $*$ con cui sono descritti i filtraggi lineari nel dominio spaziale, cioè l'applicazione di una funzione $f$ a una funzione $h$ chiamata filtro (filter kernel, descrive il sistema) per ogni valore di $x$. Gode della proprietà commutativa.
$$g * f = \int_{x=-\infty}^{\infty}g(x - s)f(s) ds$$
Nel caso di funzioni discrete, essa è definita come una somma di prodotti tra gli elementi di $f$ e i coefficienti di $h$, con funzioni che sono in realtà sequenze:
$$g(x) = \sum_{m}f(m)h(x - m)$$

\begin{wrapfigure}{R}{0.3\textwidth}
	\vspace{-15pt}
	\includegraphics[width=0.3\textwidth]{Lezioni/Immagini/convoluzione}
	\vspace{-15pt}
\end{wrapfigure}

Se i segnali non si sovrappongono, la somma dei loro prodotti è 0. Quando iniziano a sovrapporsi progressivamente, se sono positivi, i valori crescono fino a raggiungere un massimo, per poi diminuire.

La lunghezza finale della funzione ottenuta dipende dai due segnali che la compongono: assumendo che non ci siano valori nulli, con $A$ punti di $f(m)$ e $B$ di $h(-m)$ si ha $g(x)$ con $A + B - 1$ punti.

Applicando la definizione di trasformata di Fourier, si può dimostrare il fondamentale teorema della convoluzione. La trasformata della convoluzione di due funzioni è il prodotto delle trasformate delle due funzioni: 
$$G(u) = \digamma[g(x)] = \digamma[f(x) * h(x)] = F(u)H(u)$$
La somma di prodotti è più complessa computazionalmente rispetto alla trasformata, e il teorema della convoluzione permette di utilizzare quest'ultima. Ciò funziona grazie alla proprietà di traslazione. 

Per la corrispondenza tra dominio spaziale e dominio delle frequenze, si hanno le seguenti relazioni:
$$g(x) = f(x) * h(x) \Longleftrightarrow G(u) = F(u)H(u)$$
$$g(x) = f(x)h(x) \Longleftrightarrow G(u) = F(u) * H(u)$$
Un prodotto nello spazio-tempo, quindi, corrisponde a una convoluzione di trasformate. Per capire il comportamento del segnale è sufficiente osservarlo nel dominio trasformato.

La convoluzione è un operatore lineare, verificabile attraverso la relativa proprietà:
$$f(x) * [\alpha g_1(x) + \beta g_2(x)] = \alpha[f(x) * g_1(x)] + \beta[f(x) * g_2(x)]$$

\section{Quantizzazione}
La quantizzazione è un processo di discretizzazione dell'ampiezza: i segnali a tempo discreto sono convertiti a valori discret (digitali), cioé appartenenti a un insieme limitato di possibili scelte.

Risoluzione: un campione reale che necessita ipoteticamente di un numero infinito di bit per essere rappresentato, è espresso su un numero finito. Il processo è irreversibile, a causa della perdita di informazione.

La caratteristica di un quantizzatore è pertanto la curva non lineare, ma a gradini. Per tutti i valori di input che appartengono a uno degli intervalli su cui sono definiti i gradini, l'output assume il valore del gradino corrispondente.

La funzione più semplice in cui l'input è uguale all'output è la diagonale: la funzione ottenuta sarà quindi simmetrica, con una curva caratteristica non uniforme (larghezza dei gradini). Il quantizzatore è uniforme se tutti i livelli sono distribuiti ugualmente rispetto all'asse delle ascisse. La dinamica $[-V, V]$ viene quindi divisa in sottointervalli della stessa ampiezza $\Delta = 2V / L$ dove $L$ è il passo. 

Così come il passo di campionamento, esiste anche il passo di quantizzazione, cioè la distanza tra il valore minimo e massimo. L'altro parametro è la risoluzione, cioè il numero di livelli proporzionale al numero di bit (con $n$ bit, $2^n$ livelli). 

Il processo consiste nell'associare a ciascun campione $x(m)$ il numero binario $x_q(m)$ corrispondente al livello quantizzato dell'intervallo in cui case $x(m)$.

Tra tutti i livelli è necessario definire un criterio di scelta, essendo l'operazione irreversibile. Il quantizzatore è in grado di coprire solo un range $L$ di valori.

Se $L$ è maggiore del passo di quantizzazione $\Delta$, il valore massimo $L\Delta$ sarà un intervallo troppo grande rispetto alla variazione del segnale e non verrà usato completamente (spreco). \\
Al contrario, se le variazioni del segnale sono molto ampie, il range sarà grande e andrà adattato a un numero ristretto di livelli. 

Un quantizzatore è carattererizzato da una dinamica di ingresso, cioé un massimo range di valori ammissibili. Se il segnale supera gli estremi, esso viene modificato attraverso la saturazione o la saturazione con azzeramento.

L'intervallo di copertura, definito dalla risoluzione e dal passo, è quindi $D_q = \Delta L = \Delta 2^n$. Si utilizza un range dinamico per avere un numero flessibile di livelli rappresentabili, convertibile in decibel: l'obiettivo è il riempimento di tutti i livelli.

Il range dinamico è il rapporto tra il numero minimo e massimo di valori rappresentabili (espressi dal numero di bit):
$$20\log_{10} L$$
Se la risoluzione è 16 bit, il range dinamico è $20\log_{10} 2^{16} \approx 96 dB$.

Le strategie di quantizzazione sono il troncamento ($\Delta$)e l'approssimazione ($\frac{\Delta}{2}$): entrambe causano un errore, e di conseguenza lo sfasamento dei gradini rispetto alla diagonale. 

Si definisce errore (rumore) di quantizzazione la differenza tra il valore quantizzato e il valore reale del campione:
$$\varepsilon_q(n) = x_q(n) - x(n)$$
L'incidenza dell'errore nel segnale è misurata come SRN (Signal-Noise Ratio, potenza), e tipicamente si misura in dB. Il valore finale, naturalmente, non è influenzato solamente dalla quantizzazione ma anche da fattori esterni, quindi è difficile avere stime oggettive della dinamica del quantizzatore.

La qualità del segnale quantizzato si esprime come rapporto della potenza $P_S$ media del segnale a tempo discreto $x(n)$ e la potenza media dell'errore di quantizzazione $P_N$.
$$SNR_Q = 10\log_{10}\frac{P_S}{P_N}$$
Le valutazioni sono eseguite in valore assoluto, per eliminare la possibilità di annullamento dei termini. La potenza del rumore è legata al modulo della sequenza. 
$$P_N = \frac{1}{N}\sum_{n=0}^{N-1}\abs{\varepsilon_q(n)}^2 \qquad P_S = \frac{1}{N}\abs{x(n)}^2$$
Se il rumore è semplice e il segnale ha ampiezza nella dinamica del quantizzatore, varia nell'intervallo $\pm \nicefrac{\Delta}{2}$, ed è equiprobabile e casuale con valor medio nulla (distribuzione uniforme). La potenza del rumore equivale alla varianza di $\varepsilon$, variabile casuale.

La situazione ideale ha un rapporto segnale-rumore ottimale (uguaglianza), cioè esso si adatta perfettamente ai livelli e la dinamica del quantizzatore è uguale a quella del segnale. Il range di valori è pari a $2A$, per un segnale sinusoidale.
$$P_N = \sigma^2_\varepsilon = \frac{\Delta^2}{12} = \frac{A^2/3}{2^{2b}} \qquad P_S = \frac{A^2}{2}$$
$$x_a(t) = A\cos(\omega_0t) \text{ segnale sinusoidale analogico}$$
$$P_S = \frac{1}{N} \sum_{n=0}^{N-1}\abs{x(n)}^2 \text{ segnale sinusoidale campionato}$$
La distanza dei livelli, di conseguenza, sarà $\nicefrac{2A}{2^b}$, essendo la potenza del rumore strettamente legata all'ampiezza del segnale (aumenta in proporzione al numero di bit).

Conoscendo il modo di calcolare $P_S$ e $P_N$, insieme alla dinamica del quantizzatore $D_q = 2^b\Delta$, si ottiene:
$$SNR_q = 20\log\frac{\sqrt{P_S}}{D_q} + 10,8 + 6,02b$$
Il range del segnale sinusoidale è $D = 2A$, e con $b$ bit di quantizzatore in $"^b$ livelli è possibile ricavare una nuova rappresentazione sostituendo nella formula generale:
$$SNR_Q = 1,76 + 6,02b$$
Per un insieme di segnali più ampio che si distribuisce sull'intero range dinamico del quantizzatore, si somma 1,25 invece che 1.76. Un incremento di un bit porta a un incremento di circa 6 dB, cioè l'intensità del segnale raddoppia rispetto all'intensità del rumore.

Dato che la quantizzazione provoca una perdita irreversibile di informazione, tenendo conto dell'errore, è importante stabilire in quali condizioni la distorsione introdotta è minima. Anche se il campionamento è ideale, la conversione D/A può invertire solo quest'ultimo senza considerare il rumore, quindi $x_r(t) \neq x(t)$.

Maggiore è la risoluzione (bit), più il segnale si avvicina all'originale, e l'ottimo del quantizzatore è il numero minimo dei livelli per avere una distorsione accettabile.

L'operazione che permette di ricostruire il segnale analogico $x(t)$ a partire dalla sequenza $x(NT_c)$ è detta ricostruzione. Essa è fedele se:
\begin{itemize}
	\item Non c'è quantizzazione;
	\item Il campionamento è avvenuto nel rispetto del teorema di Shannon.
\end{itemize}
Nei convertitori digitale/analogico i punti del segnale vengono interpolati per generare il segnale analogico ricostruito. L'accuratezza della ricostruzione dipende dalla qualità del processo di conversione.

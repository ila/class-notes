\section{Esercizi}

\subsection{Campionamento}
\subsection{Esempio}
$$x(n) = A\cos(\omega n + \theta)$$
\begin{figure}[h]
	\centering
	\includegraphics[scale=0.56]{Lezioni/Immagini/esempio1}
\end{figure}

$\omega = 2\pi f_N$, dove $\omega$ è la pulsazione e $f_N$ la frequenza normalizzata, cioè il numero di cicli al secondo (continuo), con campioni al posto dei secondi nel mondo campionato.

Per capire la frequenza $f_N$ del segnale rappresentato, bisogna guardare quanti cicli ci sono per campioni (o quanti campioni ci sono in un ciclo, cioè prima che il pattern si ripeta) e calcolare l'inverso. In questo caso un periodo è 12 campioni, quindi $f_N = 1/12$.

La frequenza al secondo (Hz) è ottenibile sapendo quanta distanza c'è tra un campione e l'altro, altrimenti per ogni rappresentazione ci sarebbero infinite sinusoidi.

La pulsazione $\omega$ è uguale a $2\pi f_N$, quindi $\frac{2\pi}{12} = \frac{\pi}{6}$. \\
Lo sfasamento con anticipo di 2 campioni si trova effettuando la proporzione: un periodo è $2\pi$, se 12 campioni sono in$2\pi$ allora $2\pi : 12 = \theta : 2 \rightarrow \theta = \frac{\pi}{3}$.

La frequenza normalizzata indica quanto bene si vedono i campioni e la loro variazione in un periodo, per poi decidere la finestra di osservazione. La frequenza dipende dal numero di campioni al secondo ($f_c$): se la frequenza di campionamento è $f_c = 1$ campione/secondo, si ha che $f = 1/12$ ciclo/secondo, cioè $f = f_c \cdot f_N$.

Per ricavare la formula più facilmente si possono utilizzare le unità di misura. Un altro modo per arrivare allo stesso risultato è sostituire a $n\Delta t$ la $t$.

Con 2 campioni al secondo, cioè $f_c = 2$ e $\Delta t = 0.5 = \nicefrac{1}{2}$, la frequenza $f$ in cicli al secondo (Hz) è $\nicefrac{1}{2}$:
$$f_N \cdot f_c = \frac{1}{12} \frac{\text{cicli}}{\text{campione}} \cdot \frac{2 \text{ campioni}}{\text{sec}} = \frac{1}{6} \frac{\text{cicli}}{\text{sec}}$$
$$x(n) = A\cos \Big(2\pi \frac{1}{12}n + \frac{\pi}{3}\Big) \implies x(t) = A\cos\Big(2\pi \frac{1}{6}t + \frac{\pi}{3}\Big) \qquad \Delta tn \implies t$$

\subsection{Convoluzione}
Per rappresentare una convoluzione è necessario conoscere la posizione dello 0 sull'asse $x$, e rappresentare sia i campioni di $h(m)$ che di $f(m)$. Si ricorda che $\Delta m$ esiste solo quando l'argomento si annulla.

Il segnale $h(m)$ viene traslato e ribaltato rispetto all'asse $x$ con $n = -\infty, \dots, +\infty$ (anche se la proprietà è commutativa quindi il risultato sarebbe identico utilizzando $f(m)$. 

$y(n)$ è la funzione che si ottiene quando il segnale ottenuto assume valore $n$. \\
Si ha $y(n) = \sum_{-\infty}^{max+\infty} x(m)h(n - m)$.


Dato il segnale a tempo discreto $x(n) = 6,4 \cos (\nicefrac{\pi}{n})$, trovare il numero (intero) di bit necessari per la conversione da analogico a digitale tale che:
\begin{enumerate}
	\item $\Delta = 0,1$: \\
	$0,1 \cdot 2^b \geq 12,8 \rightarrow 2^b \geq \frac{12,8}{0,1} = 128 \rightarrow b = 7$ \\
	$SNR_q = 1,76 + 6,02b = 43,9 dB$
	\item $\Delta = 0,02$: \\
	$0,02 \cdot 2^b \geq 12,8 \rightarrow 2^b \geq \frac{128}{0,02} = 640 \rightarrow b = 9 \lor 10$ \\
	Il valore scelto dev'essere sempre il maggiore (10) altrimenti si verifica saturazione. \\
	$SNR_q = 1,76 + 6,02b = 61,96 dB$ se fosse ottimale;
	$SNR_q = 10\log_{10} \frac{(6,4)^2}{0,02 / 12} = 57,88 dB$ nel caso vero.
	\end{enumerate}
Bisogna sempre rispettare $D_s \leq D_q$, pertanto $D_s = 2 \cdot 6,4 = 12,8$. La dinamica del quantizzatore è $\Delta 2^b$. 

Il rapporto segnale-rumore del secondo caso è minore di quello ideale, ma comunque maggiore del primo, quindi la seconda $\Delta$ è più efficiente.

Con b = 10, 2A = 12,8, trovare la $\Delta$ tale che il rapporto segnale-rumore sia ottimo.\\
$Dq = Ds \rightarrow \Delta = \frac{12,8}{1024} = 0,0125$

Frequenza massima 15kHz, densità spettrale (potenza) uniforme: $P = \abs{F(\delta(x))}$
Il range in kHz è $[-15, +15]$ e ha una distribuzione uniforme, quindi sarà rappresentato con una finestra. 
Se la frequenza di campionamento è minore di 30, ci saranno fenomeni di aliasing e sotto-campionamento. 
Con $f_c = 27 kHz$, la finestra occupata sarà $[-13,5, 13,5]$ e la replica si posizionerà nella fascia non utilizzata in modo simmetrico rispetto all'estremo dell'intervallo. Le frequenze più alte verranno erroneamente convertite in frequenze basse.

Rapporto segnale-distorsione: applicando la definizione di rapporto segnale-rumore, è possibile calcolare l'integrale: $S = \int_{-f_c/2}^{f_c/2}\abs{x(f)}^2ndf$
$D = \int_{f_c/2}^{\infty} \abs{x(h)}^2 df +  \int_{-\infty}^{f_c/2}\abs{x(f)}^2ndf$
$SNR_c = 10\log_{10} \frac{P_R}{P_R} = \frac{27u}{3u} = 9,54 dB$

Due segnali analogici, $x(t) = 5\sin(60\pi t)$ e $y(t) = 2\cos(60\pi t) - 3\sin(40\pi t)$, vengono sommati e campionati con una frequenza di campionamento $f_c = 50 Hz$. 

Seno e coseno non hanno distinzione nella trasformata, quindi è necessario rappresentare la parte reale e la parte immaginaria nella rappresentazione. 
$60\pi t = 2\pi 30t$ quindi la parte reale avrà due picchi a $[-30, 30]$ alti 1 (2/2), mentre la parte immaginaria avrà gli stessi alti $5 / 2$, uno positivo e uno negativo.

Lo stesso procedimento si applica al seno, per ottenere il segnale prima del campionamento. C'è aliasing, perchè $50 < 2 * 30$. 

Il primo picco nella parte reale trasla fino a $+20$, e il secondo esce dall'intervallo. Il periodo base è largo $50 Hz$, e va osservato centrato in 0 con un'ampiezza di $[-25, +25]$. La seconda replica va da 30 a 30 - 50, cioè -20. La parte immaginaria ha un range $[-25, +\infty]$.

Il segnale frainteso con aliasing diventa $z'(t) = 2\cos(40\pi t) - 8\sin(40\pi t)$.

Caratterizzare ($\Delta$, numero di bit) un quantizzatore ideale per il segnale, in modo che il rapporto segnale-rumore sia circa 96 dB.

Se il segnale ha un range maggiore del quantizzatore ideale, per esempio ottenuto sommando più sinusoidali, esso satura.

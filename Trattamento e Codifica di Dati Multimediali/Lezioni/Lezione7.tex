\section{Equazioni alle differenze}
I LTI possono essere rappresantati in modi differenti, tra cui la combinazione lineare con coefficienti costanti nel tempo (invariante). 

Accumulatore: memoria infinita, generalmente si assegna un valore all'istante di partenza. 

Il ritardo sulla y indica di che ordine è il sistema: il primo ordine, per esempio, ha ricorsione con un campione in ritardo. 

La media cumulativa è un sistema ricorsivo non stazionario, con normalizzazione effettuata in base a un parametro non costante. 

La radice quadrata non è un sistema lineare: non esistono sommatorie, ma il termine y compare al denominatore quindi non è rappresentabile come combinazione lineare. 

La risposta all'impulso viene descritta attraverso la convoluzione, come sequenza finita o infinita. I primi non hanno la ricorsione, sono campioni combinati con pesi: i valori di h sono i coefficienti di x (bk). \\
Se la sommatoria va a infinito, non c'è una diretta corrispondenza con il dominio delle frequenze, quindi è inutile cercare y(n) guardando la risposta ma si utilizza il dominio trasformato.
\section{DCT wavelets}
L'obiettivo dell'analisi è la riduzione delle ridondanze, spostandosi in uno spazio dove le informazioni e i canali sono separati. 

Nel dominio diretto le componenti di un segnale $X$ sono tra loro significativamente correlate, quindi la stessa informazione è ridondante. Spostandosi nel dominio trasformato $Y = T[X]$, si cerca una rappresentazione dove le componenti siano molto meno correlate.

$Y$ può essere codificato in modo più efficiente di $X$, utilizzando solo le componenti del segnale che descrivono l'informazione. Lo spettro, per esempio, ha un contributo maggiore espresso con le basse frequenze al centro, quindi è possibile codificare solo una porzione determinata di energia.

In termini di compressione, sono necessari solo i bit delle componenti da tenere. Il modello di compressione permette di quantizzare con perdite trascurabili, in un cambio di spazio secondo due strategie:
\begin{itemize}
	\item Codificatore di sorgente, che riduce le ridondanze;
	\item Codificatore di canale, che incrementa l'immunità al rumore.
\end{itemize}

\subsection{Codifica con trasformate}
Questo algoritmo introduce perdita ed è computazionalmente costoso, a causa del calcolo delle trasformate. Una trasformata lineare e reversibile è usata per il mapping dell'immagine in un set di coefficienti che vengono poi quantificati e codificati.

Il mapping può essere effettuato secondo diverse metodologie, a seconda della capacità di decorrelazione dei dati, semplicità di realizzazione e altri fattori.

La KLT (analisi alle componenti principali) è la trasformata ottima, ma è computazionalmente inefficiente. DCT, invece, approssima il comportamento ottimo, ed è usata negli algoritmi di codifica più utilizzati (jpeg, mpeg).

La trasformata wavelet permette la multirisoluzione (jpeg2000), e funziona secondo il principio di determinazione di Heisenberg, trovando un compromesso tra le frequenze e il dominio temporale lavorando con tradeoff.

Data una funzione $f(x)$, la sua DFT $F(u)$ è:
$$F(u) = \frac{1}{M} \sum_{x=0}^{M- 1}f(x)e^{-j\frac{2\pi}{M}ux} = \frac{1}{M} \sum_{x=0}^{M- 1}f(x)g(u, x)$$
$$g(u, x) = e^{-j\frac{2\pi}{M}ux}$$

La trasformata è lineare e invertibile, e $g(u, x)$ è detto kernel della trasformazione diretta, da cui dipendono le proprietà. I kernel devono essere invertibili, lineari e separabili.

Trasformata di Fourier discreta 1D:
$$e^{-j\frac{2\pi}{M}ux} = \cos\big(\frac{2\pi}{M}ux\big) + j\sin\big(\frac{2\pi}{M}ux\big)$$
% immagine onde quadrate
Il segnale di partenza è proiettato in base al suo tempo e spazio con le sue parti reale e immaginaria. Seno e coseno hanno simmetria rispettivamente dispari e pari, e trasformando si osserva l'influenza di ciascuna componente. 

La trasformata coseno discreta (DCT) è lineare, con kernel diretto uguale a quello inverso (simmetria speculare), separabile e simmetrico. La componente continua è legata al valore medio dell'immagine (0, 0). 
$$T(u) = \sum_{x=0}^{M- 1}f(x)g(u, x) \qquad g(x, u) = \alpha(u) \cos\Big[\frac{(2x + 1)\pi u}{2N}\Big]$$
% immagine coseno
$$\alpha(u) = \begin{cases}
\sqrt{\frac{1}{N}} & u = 0 \\
\sqrt{\frac{2}{N}} & u = 1, 2, \dots, N - 1^{}
\end{cases}$$

La separabilità permette di calcolare la trasformata 2D tramite applicazioni successive della trasformata 1D alle righe e alle colonne, senza perdita di informazione. Ciascun blocco è costituito da $N \times N$ sottoblocchi.

Una maggiore quantità di informazione è presente nei primi coefficienti della DCT, rispetto allo stesso numero di coefficienti della DFT.
% dct vs dtf
Visualizzando un'immagine su un monitor con un range elevato (scala logaritmica), non si vede nulla perché il monitor arriva solo fino a 256.

\subsection{Analisi multirisoluzione}
Le immagini sono generalmente costituite da regioni connesse che formano gli oggetti, omogenee rispetto a una qualche proprietà.

L'analisi multirisoluzione permette di mettere in evidenza sia le immagini a bassa frequenza che quelle ad alta, con segnali e campioni di dimensioni diverse, concentrandosi su differenti posizioni nello spettro. 

Le caratteristiche locali di un'immagine sono contraddistinte da variazioni statistiche locali, dovute  discontinuità fra regioni omogenee. Caratteristiche nascoste a una data risoluzione possono essere individuabili a un'altra.

La tecnica più frequente è l'analisi piramidale, che parte dalla risoluzione massima e progressivamente sottocampiona, evidenziando i cambiamenti tra i dettagli e le perdite tra un livello e l'altro ricostruendo ogni volta l'immagine con meno dettagli. I pixel mancanti vengono approssimati tramite wavelet.

Il segnale è decomposto in un insieme di sottosegnali (sottobanda, analisi), passando attraverso filtri complementari, ciascuno dei quali agisce su una fascia di frequenze. Ricampionando, filtrando e ricombinando le sottobande (sintesi) si ottiene un'approssimazione $\hat{x}(n)$ dell'originale, non incorrendo in aliasing. 

Ciascuna sottobanda $(y_0(n), y_1(n))$ è ottenuta filtrando con un passa-banda $(h_0(n), h_1(n))$ il segnale originale. Poiché la sottobanda ha spettro limitato e pari a metà dell'originale, è possibile sottocampionare senza perdita di informazione. 
% immagine
%% da qui
Imponendo ..., si trovano le coppie di filtri di sintesi e corrispondenti filtri di analisi che garantiscono una perfetta ricostruzione. In caso di più dimensioni, possono esserci $n$ filtri per righe e colonne che mettono in evidenza diverse porzioni di frequenze rispetto alla direzione. 

Dato che le statistiche locali sono facilmente modellabili e presentano molti valori nulli, questa codifica è particolarmente vantaggiosa per la compressione: non tutto l'istogramma viene occupato, oppure ci sono poche alte frequenze che possono essere eliminato. L'approccio di denoising per ridurre il rumore è utilizzato nelle applicazioni mediche.

Il filtro passa-basso è una funzione di scala, mentre il passa-alto è una wavelet generata a partire da una wavelet madre. La formula in notazione monodimensionale è:
$$f(x)  = \frac{1}{den\sqrt{m}} \sum_{k} W_\varphi (j_0, k) \varphi_{j_0, k} (x) + \frac{1}{den\sqrt{m}} \sum_{j=j_0}^{J}\sum_{k} W... $$
Trasformando con la wavelet, è possibile pesare ogni contributo per capire quali frequenze e bande hanno influenza maggiore, valutando le features che caratterizzano l'immagine. 

Per calcolare gli edge, si ha il segnale di partenza in una somma di contributi: la parte interessata viene tenuta applicando la formula, mentre le altre componenti vengono annullate. 

Un altro approccio è il denoising, a partire dalle wavelet. La ricostruzione è effettuata dopo aver posto una soglia alta sui coefficienti a tutte le risoluzioni, accettando solo i valori al di sopra. 

\section{Tecniche di compressione}
Le tecniche di compressione si dividono in due grandi famiglie: lossy (con perdita accettabile a seconda dell'applicazione) e lossless (senza perdita). I dati ridondanti vengono rimossi, e il risparmio viene misurato tramite un rapporto di compressione in bit, dipendente appunto dalla ridondanza. 

Un tipo di ridondanza è statistico: i vicini possono essere correlati o dipendenti, quindi una parte dell'informazione è ripetuta. Alcuni livelli secondo l'istogramma hanno una maggiore probabilità di essere occupati, quindi è possibile calcolare il numero medio normalizzato di bit necessari per ogni frequenza. 

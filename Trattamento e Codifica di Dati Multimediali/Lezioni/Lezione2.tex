\section{Segnali}
Un segnale rappresenta il comportamento di grandezze fisiche in funzione di una o più variabili indipendenti. Sono monodimensionali se rappresentati da una sola variabile, per esempio il suono (continuo). I dati EEG sono multidimensionali in variazione al tempo, agli elettrodi e ai soggetti.

Le immagini in bianco e nero sono segnali bidimensionali (coordinate spaziali) e monocanale (il grigio), mentre quelle a colori hanno 3 segnali dimensionali RGB. Il campionamento corrisponde al numero di pixel, e la quantizzazione è la profondità del colore (quanti bit per la codifica). Aumentando il numero di livelli, aumenta la capacità di rappresentare l'informazione.

Se la variabile indipendente continua viene discretizzata è stato effettuato un campionamento, in cui è necessario conoscere la distanza tra i campioni (digitali). 

Il valore assunto dal segnale si definisce ampiezza (dipendente, codominio) mentre l'asse delle ascisse è il dominio (tempo o spazio). Si possono introdurre grandezze statistiche come media e varianza, indicate in modo diverso a seconda del tipo di segnale.

\begin{itemize}
	\item Continuo:
	\begin{itemize}
		\item $\mu = \lim\limits_{T \rightarrow \infty} \frac{1}{T} \int_{-\frac{T}{2}}^{\frac{T}{2}} x(t) dt$;
		\item $\mu = \frac{1}{T_1 - T_0} \int_{T_0}^{T_1} x(t) dt$;
	\end{itemize}
	\item Discreto:
	\begin{itemize}
		\item $\mu = \frac{1}{N} \sum_{i = 0}^{N - 1} x_i$.
	\end{itemize}
\end{itemize}

Il segnale digitale ha solamente una casistica di $\mu$ perché non tende mai a infinito, e il livello dell'ampiezza è diverso. Se un segnale varia ha una componente continua DC (direct current), contributo a frequenza 0 e valore medio, e componenti AC a corrente alternata che variano in base a come il segnale fluttua intorno al valore medio.

Una forma d'onda ripetuta ha escursioni costanti, descritte da una grandezza chiamata ampiezza picco-picco $A_{pp}$. Il periodo è arbitrario.

Deviazione standard e varianza forniscono informazioni aggiuntive su quanto lontano (e con quale potenza) il segnale fluttua dal valore medio. Alla varianza è fortemente legata la potenza del rumore: un'alta varianza implica un forte rumore. \\
$\sigma^2 = \frac{1}{N - 1} \sum_{i = 0}^{N - 1} (x_i - \mu)^2$

\begin{figure}[h]
	\centering
	\includegraphics[scale=0.48]{Lezioni/Immagini/mediavarianza}
\end{figure}

Se media e varianza di un segnale non cambiano nel tempo, esso è stazionario. Al contrario, se il segnale varia la media sarà diversa a seconda della finestra, e la media globale non dà informazioni.

La \textbf{periodicità} indica la ripetizione del segnale nel tempo, definito appunto in periodi. Non esistono segnali puramente periodici, ma si usano approssimazioni delle forme d'onda che assume il segnale. L'inversa del periodo è chiamata \textbf{frequenza fondamentale}: $f_0 = 1/T$. 

\subsection{Segnale sinusoidale}
Un segnale sinusoidale è un seno o un coseno, monodimensionale in funzione del tempo o dello spazio.

$$A(T) = A_{med} + B \cdot \sin(2\pi ft + \varphi_0)$$
$$A_{med} = \frac{1}{T} \int_{0}^{T} A(t) dt \qquad A_{pp} = A_{max} - A_{min} \qquad B = A_{pp} / 2$$

Parametri importanti in questi casi sono frequenza e fase. \\
La frequenza si misura in Hertz, e rappresenta la rapidità con cui varia l'ampiezza in un intervallo temporale $T$. La pulsazione (intera variazione di ampiezza) è proporzionale alla frequenza, si ha che $\omega = 2\pi f$. \\
La fase segna l'alternarsi di positività o negatività del segnale, in particolare è significativa la fase iniziale $\varphi_0$.

$$P(t) = \abs{x(t)}^2 \qquad \text{potenza istantanea}$$
$$E_x = \int_{-\infty}^{+\infty} P(x) dt \qquad \text{energia: area di potenza istantanea}$$

Tanto più l'ampiezza si scosta dallo 0, la potenza aumenta. Se $E_x < \infty$ il segnale ha energia finita. \\
Quando $E_x = \infty$ si definisce la potenza media: $P_x = \lim\limits_{T \rightarrow \infty} \frac{1}{T} \int_{-\frac{T}{2}}^{\frac{T}{2}} \abs{x(t)}^2 dt$ \\
Potenza media di un segnale periodico $T$: $P_x = \frac{1}{T} \int_{T} \abs{x(t)}^2 dt$

Entrambi i valori hanno una componente continua (valore medio) calcolato come il limite dell'integrale o l'integrale (segnale periodico) di $x(t)$.

\subsection{Decibel}
Il \textbf{decibel} dB è un'unità di misura logaritmica (quindi non lineare), di cui lo scopo del logaritmo è visualizzare meglio grandi scale di valori e avvicinarsi alla percezione umana. La misura è quindi relativa e adimensionale.

$$deciBel = dB \longrightarrow 10\log_{10} \frac{P_1}{P_2} = 20\log_{10} \frac{A_1}{A_2}$$
$$ Bel \longrightarrow \log_{10} \frac{P_1}{P_2} \qquad P = \text{potenza} \qquad A = \text{ampiezza} \qquad P \propto A^2$$

Le componenti della formula sono due pressioni, di cui il numeratore è la potenza del suono e il denominatore è la soglia minima di udibilità. La potenza è proporzionale al quadrato dell'ampiezza, quindi trasformando in scala logaritmica essa diventa un fattore moltiplicativo. 6dB rappresentano un raddoppio dell'ampiezza.

\subsection{Trasformazioni di segnali}
Operazioni molto comuni sono la traslazione, il cambio di scala e l'inversione temporale. 
\begin{itemize}
	\item Ritardo: fissato un tempo $t_0$, la traslazione trasforma il segnale $x(t)$ nel segnale $x(t - t_0)$;
	\item Anticipo: fissato un tempo $t_0$, la traslazione trasforma il segnale $x(t)$ nel segnale $x(t + t_0)$;
	\item Cambio di scala: fissato un numero reale $a > 0$, la scalatura trasforma il segnale $f(t)$nel segnale $f(at)$;
	\begin{itemize}
		\item Se $a > 1$ si ottiene una compressione lineare;
		\item Se $a < 1$ si ottiene un allungamento lineare;
	\end{itemize}
	\item Inversione: trasforma il segnale $f(t)$ nel segnale $f(-t)$.
\end{itemize}

Se il segnale è reale, è possibile ritardarlo ma non anticiparlo. Un segnale si dice pari se $f(t) = f(-t)$, dispari se $f(t) = -f(-t)$: il coseno è una funzione pari, il seno è dispari.

\subsection{Segnali continui}
Gradino: usato per selezionare la parte positiva dei segnali che tendono a $\pm \infty$. Si definisce un gradino \textbf{unitario} $u(t)$:
$$u(t) = \begin{cases}
1 & \text{se } t \geq 0 \\
0 & \text{se } t < 0
\end{cases}$$

Gradino \textbf{traslato} in $t_0$: definito quando la finestra di osservazione è finita, centrata rispetto a 0. Se il segnale è fuori dall'intervallo assume valore 0:
$$u(t - t_0) = \begin{cases}
1 & \text{se } t \geq t_0 \\
0 & \text{se } t < t_0
\end{cases}$$

Impulso rettangolare unitario $rect(t)$:
$$rect(t) = \begin{cases}
1 & \text{se } \abs{t} \leq \nicefrac{1}{2} \\
0 & \text{se } \abs{t} > \nicefrac{1}{2}
\end{cases}$$

Quest'ultima è una funzione che forma un rettangolo di area unitaria. Generalizzando, si ha un rettangolo di altezza $A$, base $T$ e traslato in $t_0$ sostituendo nella formula unitaria $t$ con $\frac{t - t_0}{T}$ e confrontandolo con $\nicefrac{1}{2}$.
$$\abs{t - t_0} \leq T / 2 \begin{cases}
\text{per } (t - t_0) > 0 \qquad (t - t_0) \leq T / 2 \qquad t \leq t_0 + T / 2 \\
\text{per } (t - t_0) < 0 \qquad (t - t_0) \geq -T / 2 \qquad t \geq t_0 - T / 2
\end{cases}$$

La moltiplicazione di un segnale per un rettangolo lo approssima con segmenti verticali o orizzontali a seconda dell'asse considerato. Media e varianza non sono le stesse rispetto alla funzione originale.

Funzione \textbf{delta di Dirac} $\delta(t)$: distribuzione con rettangolo di base infinitesima e altezza infinita che abbia l'area unitaria, con la larghezza che tende a 0 e di conseguenza l'altezza che tende a infinito. 
$$\int_{-\infty}^{\infty} \delta(t) dt = 1$$
Limite dell'impulso rettangolare di base $\Delta$ per $\Delta \rightarrow 0$: 
$$\delta(t) = \lim\limits_{\Delta \rightarrow 0} \frac{1}{\Delta} \text{ rect}\Big(\frac{t}{\Delta}\Big) \qquad \delta(t - x) = 0 \qquad \text{se } t \neq x$$
Viene introdotta per rappresentare fenomeni fisici di durata infinitesima (impulsi).

discreto: delta di Kronecker o impulso unitario, freccia verticale. IMPORTANTE
n-2 = delta in ritardo di 2, + anticipo

Il gradin o continuo nel discreto diventa una successione di delta. 
u(n) = sommatoria k=0 to inf delta(n-k)

\section{Sequenze}
causali: n positivi
anticausali: n negativi
= considerare i campioni da adesso in poi causali

diagonale = funzione rampa, fattore moltiplicativo n

\section{Analisi di Fourier}
Decompone il segnale in costituenti sinusoidali di differenti frequenze. Il segnale non è più nel dominio tempo-spazio, ma delle frequenze: i dati sono gli stessi, cambia solo la rappresentazione.

 








